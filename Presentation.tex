
\documentclass[slidestop]{beamer}

\usepackage{beamerthemesplit}
\usepackage{mathpartir}
\usepackage{amsthm}
\usepackage{graphicx}
\usepackage{grffile}
\usepackage{graphics}
\usepackage{pstricks}
\usepackage{bytefield}
\usepackage{listings}

\lstset{language=C}
\lstset{basicstyle=\ttfamily, columns=flexible}


\newcommand\Kwd[1]{{\sffamily\bfseries{#1}}}
\newcommand\EOf[4]{{#1}\vdash_{#2}{#3}:{#4}}
\newcommand\MOf[4]{{#1}\vdash_{#2}{#3}\div{#4}}
\newcommand\IsProc[3]{{#1}\vdash_{#2}{#3}\ \mathsf{proc}}
\newcommand\IsAction[3]{{#1}\vdash_{#2}{#3}\ \mathsf{action}}

\title{Implementing DNS in C0}
\author{Jonathan Sterling \and Klaas Pruiksma \and Farzaneh Derakhshan}


\begin{document}

\frame{
   \begin{center}
    \huge{Implementing DNS in C0}\\
    \vspace{48pt}
    \Large{15-712 Project}\\
    \vspace{24pt}
    \large{\today}
  \end{center}
}

\frame{\frametitle{Background-DNS}
 \begin{center}

  {\bf \red Map domain names to IP addresses in a distributed manner. Consists of: }
 \end{center} 
\begin{itemize}
\item Domain name space with reource records, a tree-like data structure with a root, and nodes labeled with at most 63 characters.
\item Name servers keep information about a complete sub-tree of the domain name. They may also cache pointers to other name servers.
\item Resolvers that provide an interface between applications and DNS in the process of name resolution
\end{itemize}
}
\frame{\frametitle{DNS-Name Resolution}
\begin{itemize}
\item A user program sends a name query to the resolver of the DNS client.

\item The DNS resolver looks up the local domain name cache for a match. If a match is found, it sends the corresponding IP address back. If not, it sends a query to the DNS server.

\item The DNS server looks up the corresponding IP address of the domain name in its DNS database. If no match is found, it sends a query to a higher level DNS server. This process continues until a result, whether successful or not, is returned.

\item The DNS client returns the resolution result to the application after receiving a response from the DNS server.
 \end{itemize}
}
\frame{\frametitle{Background-C0}
 \begin{center}

  {\bf \red C0 programming language is a restricted subset of C, that provides:}
\begin{itemize}
\item Implicit {\bf memory safety}.\\
\item Restrictions on {\bf pointer arithmetic} and {\bf casting}: 
\begin{itemize}
\item prevent any attempt to access an array out of its
bounds.\\
\end{itemize}
\item {\bf Garbage-collection} 
\begin{itemize} 
\item no need for error-prone manual memory management.\\
\end{itemize}
\item {\bf Contracts}: dynamically checks correctness conditions for programs\\
\end{itemize}
 \end{center}
}

\frame{\frametitle{Related Work-Nail}
{\bf \red Nail}: A specialized language for specifying low-level data formats, have been used to develop a {\bf \red “authoritative DNS server”.} \vspace{12pt}

 The most error-prone aspects are out of Nail's scope:\\
\begin{itemize}
\item {\bf Internal implementation of Nail} written in {\bf C}. 
\begin{itemize}
\item “Semantics” of C code is informal and indefinite.
\end{itemize}
\item {\bf User-provided implementation} of compression and decompression functions. 
\begin{itemize}
\item An operator precedence bug in the Nail's DNS demo .
\end{itemize}
\end{itemize}
\vspace{12pt}

Conntrary to Nail, we code parsers and formatters {\bf manually}, but in a {\bf memory-safe} and {\bf type-safe language}.

\begin{center} 
{\bf \red While functional correctness is not guaranteed, safety is guaranteed.}
 \end{center}
}
\frame{\frametitle{Related Work-Fox}
{\bf \red Fox }: Implementation of the DNS protocol in a safe and formally specified language,
\begin{itemize}
\item undertaken at Carnegie Mellon University in the 1990s.\\
\item The FoXNet library implemented

\begin{center} An array of low-level network infrastructure (TCP, IP, ARP and DNS) in the Standard ML language.\end{center}
\item Unlike our development, FoxNet had a highly modular character.
\end{itemize}
}


\frame{\frametitle{Our Project}
 \begin{center}
  {\bf \red Developing an implementation of the DNS protocol in the memory-safe C0 language}
 \end{center}
Two principal components:
\begin{itemize}
\item Network Interface
\item Marshalling and Unmarshalling
\end{itemize}

}

\frame{\frametitle{Network Interface}
C0 has a very limited standard library.
 \begin{itemize}
   \item We write a networking library for C0.\vspace{12pt}
   \item Due to limitations of the language, This library cannot be written entirely in C0.\vspace{12pt}
\item We write an interface between C0 code and the C sockets library with a collection of small functions in C.\vspace{12pt}
\item Add on top of that, the C0 portion of the library, to provide a higher-level interface for network communication.\vspace{12pt}
  \end{itemize}
\vspace{12pt}}
\frame{\frametitle{Network Interface-Cont'd}
\begin{center}
{\bf \red ensure that any memory safety violations occur outside of our project}
\end{center}
\begin{itemize}
\item  Keep the C portion limited in scope and size:
\begin{itemize}
\item  By checking that there are no calls to malloc or similar functions.
\end{itemize}
\item Any accesses to C0-allocated memory are done so safely.(less easily checked)
\begin{itemize}
\item By using a collection of functions for safe memory access provided to libraries by the C0 runtime.
\end{itemize}
\end{itemize}
}
\frame{\frametitle{Network Interface-Cont'd}  
\begin{center}
limited in the scope to DNS, so we
\end{center}
\begin{itemize}
\item Focus on the functions for datagram-based sockets,
\item Do not expose the full functionality of the sockets library, 
\item A limited interface to the poll function of the C standard library in the same way to avoid busy-waiting.\\
\end{itemize}
\begin{center}
Sufficient to implement our nameserver\\
Additional library functionality can be added using these methods.
\end{center}
}

\frame{\frametitle{Marshalling and Unmarshalling}

 \begin{itemize}
   \item C0 has only a single type (32-bit) integers.\vspace{12pt}
   \item The requests and responses to the network interface formatted as arrays of 32-bit integers.\vspace{12pt}
\item The marshalling-and-unmarshalling module unpack this data into structured representations of core DNS datatypes.\vspace{24pt}
  \end{itemize}}
\frame{\frametitle{Marshalling and Unmarshalling-cont'd}
This figure shows the bit-level layout of DNS message headers:
\begin{figure}
  \centering
  \begin{bytefield}[bitwidth=1.5em]{16}
    \wordbox{1}{\texttt{ID}}
    \\
    \bitbox{1}{\texttt{QR}}
    \bitbox{4}{\texttt{Opcode}}
    \bitbox{1}{\texttt{AA}}
    \bitbox{1}{\texttt{TC}}
    \bitbox{1}{\texttt{RD}}
    \bitbox{1}{\texttt{RA}}
    \bitbox{3}{\texttt{Z}}
    \bitbox{4}{\texttt{RCODE}}
    \\
    \wordbox{1}{\texttt{QDCOUNT}}
    \\
    \wordbox{1}{\texttt{ANCOUNT}}
    \\
    \wordbox{1}{\texttt{NSCOUNT}}
    \\
    \wordbox{1}{\texttt{ARCOUNT}}
  \end{bytefield}
  \caption{The layout specification of message headers in
    DNS.}\label{fig:layout-message-header}
\end{figure}}

\begin{frame}[fragile]\frametitle{Marshalling and Unmarshalling}
 How we code the header structure in our $C0$ implementation:{\tiny
\begin{figure}
  \begin{lstlisting}[frame=single]
struct header {
  int id;
  int qr;
  int opcode;
  int aa;
  int tc;
  int rd;
  int ra;
  int z;
  int rcode;
  int qdcount;
  int ancount;
  int nscount;
  int arcount;
};
typedef struct header* header;
  \end{lstlisting}

  
\caption{Example DNS data structures in our C0 implementation.}\label{fig:c0-data-structures}
\end{figure}}
\end{frame}
\begin {frame}[fragile]\frametitle{Marshalling and Unmarshalling}
\begin{itemize}
\item To convert between these two we implement functions of the following kind:
\end{itemize}
{\small
\begin{lstlisting}
domain_name parse_domain_name(cursor cur, int[] data);
resource_record parse_resource_record(cursor cur, int[] data);
header parse_header(cursor cur, int[] data);
\end{lstlisting}}

\end{frame}
\frame{\frametitle{Marshalling and Unmarshalling-Cont'd}
\begin{itemize}
   \item To enable a less error-prone style of coding than manual cursor-passing, we implemented {\bf an abstract type of imperative
cursors}.
\item A cursor can be advanced or retreated by a certain number of bits. \\
\item Then, our low-level bit arithmetic utilities can be used to read (or write) a certain number of bits at a certain offset from (resp. to) our stream of words.\\
\item cursors are bit-indices rather than indices into the 32-bit cells of the message.
\item Because specific parts of DNS data structures may only be a few bits long, so their parsers cannot move the cursor forward a full word’s length.
  \end{itemize}
We have implemented marshalling and unmarshalling for a sizable fragment of the DNS protocol, omitting only some rare or experimental resource record types.\\
Our parser for domain names supports compression\\
During formatting we do not attempt to re-compress domain names. 
}


\frame{\frametitle{Evaluation-correctness and safety}

\begin{center}
we enjoy certain safety guarantees that are either impossible or intractable to achieve for mainstream C-based implementations.
\end{center}

 \begin{itemize}
 \item writing the majority of our DNS server in the memory- and type- safe C0 language, 
\item In return, we have sacrified some performance. 
\begin{itemize}
\item Hard to tell how much of our performance trade-off was due to garbage collection vs. the unoptimized character of our implementation
\end{itemize}
  \end{itemize}
\vspace{12pt}}
\begin{frame}[fragile]\frametitle{Evaluation-correctness and safety-Contracts}
 invaluable notion of {\bf contract}:\\
Here, we used contracts to ensure that we would not suffer from bit shift range
errors. 
a lightweight form of assertion which wraps the input and output boundaries of procedures.\\
{\small \begin{lstlisting}
int get_mask(int index, int size)
//@requires size < 32 && size >= 0;
//@requires index < 32 && index >= 0;
{
    return ((1 << size) - 1) << index;
}
int get_bits(int index, int size, int data)
//@requires size < 32 && size >= 0;
//@requires index < 32 && index >= 0;
{
    return (data & get_mask(index, size)) >> index;
}


\end{lstlisting}}

 
\end{frame}

\frame{\frametitle{Evaluation-correctness}

  The code did not begin with these contracts, but in the course of debugging,
these contracts were introduced; in every instance, introducing contracts (shifting
blame closer and closer to call-sites) enabled us to easily determine and liquidate bugs.\vspace{12pt}
   \begin{itemize}
   \item The C0 compiler can be instructed to check contracts dynamically during execution,
or to generate code which ignores them. This is helpful, since it is important to have
debuggable errors when developing, but in production one may not wish to incur the
cost of dynamic contract checks. While dynamic execution of contract checks is already
very useful, we feel that more advanced techniques such as abstract interpretation and
its special case, symbolic execution, would be even more useful.\vspace{12pt}
  \end{itemize}
}
\frame{\frametitle{Evaluation-Performance}

}


\end{document}
