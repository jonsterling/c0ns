\documentclass{article}
\usepackage[tt=false]{libertine}
\usepackage{mathpartir}
\usepackage{amsthm}
\usepackage[colorlinks=true]{hyperref}
\usepackage{bytefield}
%\usepackage{fullpage}
\usepackage{listings}

\lstset{language=C}
\lstset{basicstyle=\ttfamily, columns=flexible}


\usepackage{sectsty}
\allsectionsfont{\sffamily}

\title{Implementing DNS in C0}
\author{Farzaneh Derakhshan, Klaas Pruiksma, Jonathan Sterling}

\newtheorem{theorem}{Theorem}[section]
\newtheorem{corollary}{Corollary}[theorem]
\newtheorem{lemma}[theorem]{Lemma}

\newcommand\Lang[1]{{\sffamily\bfseries{#1}}}
\newcommand\LangCA{\Lang{CA}}
\newcommand\LangMA{\Lang{MA}}
\newcommand\EOf[4]{{#1}\vdash_{#2}{#3}:{#4}}
\newcommand\MOf[4]{{#1}\vdash_{#2}{#3}\div{#4}}
\newcommand\IsProc[3]{{#1}\vdash_{#2}{#3}\ \mathsf{proc}}
\newcommand\IsAction[3]{{#1}\vdash_{#2}{#3}\ \mathsf{action}}


\begin{document}

\maketitle

\begin{abstract}
  Please write an abstract.
\end{abstract}

\section{Background and Related Work}

\subsection{What is DNS?}

Jon

\subsection{C0 and Concurrent C0}\label{sec:c0}

Farzaneh.

\subsection{Related Work}
Jon.

\section{Our Project}

Our original project proposal was to develop a language for safe
concurrent programming with precise formal guarantees; however, the
negotiation process led us to choose a completely unrelated project,
which we describe in this section. We will be developing an
implementation of the DNS protocol in the memory-safe \Lang{C0}
language (see \S\ref{sec:c0}); we have divided our work into two
principle components.

\subsection{Network interface}\label{sec:network-interface}

Klaas.

\subsection{Marshalling and unmarshalling}

Unlike \Lang{C}, the \Lang{C0} language has only a single type of
integers, which are represented internally as 32-bit integers. The
network interface (see \S\ref{sec:network-interface}) will deal with
requests and responses formatted as arrays of 32-bit integers; it is
the responsibility of the marshalling-and-unmarshalling module to
unpack this data into structured representations of core DNS
datatypes.

\begin{figure}
  \centering
  \begin{bytefield}[bitwidth=1.5em]{16}
    \wordbox{1}{\texttt{ID}}
    \\
    \bitbox{1}{\texttt{QR}}
    \bitbox{4}{\texttt{Opcode}}
    \bitbox{1}{\texttt{AA}}
    \bitbox{1}{\texttt{TC}}
    \bitbox{1}{\texttt{RD}}
    \bitbox{1}{\texttt{RA}}
    \bitbox{3}{\texttt{Z}}
    \bitbox{4}{\texttt{RCODE}}
    \\
    \wordbox{1}{\texttt{QDCOUNT}}
    \\
    \wordbox{1}{\texttt{ANCOUNT}}
    \\
    \wordbox{1}{\texttt{NSCOUNT}}
    \\
    \wordbox{1}{\texttt{ARCOUNT}}
  \end{bytefield}
  \caption{The layout specification of message headers in
    DNS.}\label{fig:layout-message-header}
\end{figure}


\begin{figure}
  \centering
  \begin{bytefield}{16}
    \wordbox[lrt]{1}{\texttt{NAME}}\\
    \skippedwords\\
    \wordbox[lrb]{1}{}\\
    \wordbox{1}{\texttt{TYPE}}\\
    \wordbox{1}{\texttt{CLASS}}\\
    \wordbox{2}{\texttt{TTL}}\\
    \wordbox[lrt]{1}{\texttt{RDATA}}\\
    \skippedwords\\
    \wordbox[lrb]{1}{}
  \end{bytefield}
  \caption{The bit layout specification of resource records in
    DNS.}\label{fig:layout-resource-record}
\end{figure}

\begin{figure}
  \begin{lstlisting}[frame=single]
struct header {
  int id;
  int qr;
  int opcode;
  int aa;
  int tc;
  int rd;
  int ra;
  int z;
  int rcode;
  int qdcount;
  int ancount;
  int nscount;
  int arcount;
};

typedef struct header* header_t;
  \end{lstlisting}

  \begin{lstlisting}[frame=single]
struct domain_name {
  string label;
  struct domain_name* next;
};

typedef struct domain_name* domain_name_t;

struct resource_record {
  domain_name_t name;
  int type;
  int class;
  int ttl;
  int rd_length;
  int[] rdata;
};

typedef struct resource_record* resource_record_t;
  \end{lstlisting}
  \caption{Example DNS data structures in our \Lang{C0} implementation.}\label{fig:c0-data-structures}
\end{figure}

For instance, Figure~\ref{fig:layout-message-header} shows the
bit-level layout of DNS message headers, and
Figure~\ref{fig:layout-resource-record} shows the same for resource
records; Figure~\ref{fig:c0-data-structures} shows how we code these
structures in our \Lang{C0} implementation. To convert between these
structured representations and the sequences of words which are used
at the boundaries of the network
interface~(\S\ref{sec:network-interface}), we implement functions of
the following kind:

\begin{lstlisting}
int parse_domain_name(int cursor, int[] msg, domain_name_t dest);
int parse_resource_record(int cursor, int[] msg, resource_record_t dest);
int parse_header(int cursor, int[] msg, header_t dest);
\end{lstlisting}

These functions take as an argument a cursor (an index in bits into
the message), an array of 32-bit words (the entire input message), and
pointer to a structure into which to write the parsed data; they
return a new cursor. It is important for the modularity of our
development that cursors represent not indices into the 32-bit cells
of the message, but rather bit-indices; this is because specific parts
of DNS data structures may only be a few bits long, so their parsers
cannot move the cursor forward a full word's length.
%
A sample of our parsing code can be found in the supplementary
appendix to this proposal (\S\ref{appendix:decompression}). Of
course, we will also implement the inverses to the above functions.



\section{Evaluation Plan}

Klaas.


\nocite{willsey-prabhu-pfenning:2016, biagioni-harper-lee-milnes:1994, biagioni-harper-lee:2001, rfc:1034, rfc:1035}
\bibliographystyle{abbrv}
\bibliography{bibtex-references/refs,project}

\clearpage
\appendix
\section{Decompressing Domain Names}\label{appendix:decompression}

\begin{lstlisting}
int parse_domain_name(int cursor, int[] data, domain_name* dest) {
  int* len = alloc(int);

  int orig_cursor = cursor;
  cursor = read_bits(cursor, data, 8, len);

  if (*len == 0) {
    // case: done
    *dest = domain_nil();
    return cursor;
  }

  if (*len >= 0xc0) {
    // case: pointer
    int* ptr = alloc(int);
    // read the remainder of the first 16 bits as a pointer to an offset in octets
    cursor = read_bits(orig_cursor + 2, data, 14, ptr);
    // read the last label of the domain name from the pointer:
    parse_domain_name(*ptr * 8, data, dest);
    return cursor;
  }

  // case: ordinary label
  // now '*len' specifies the number of octets in the label
  char[] chars = alloc_array(char, *len);
  for (int i = 0; i < *len; i++) {
    int* octet = alloc(int);
    cursor = read_bits(cursor, data, 8, octet);
    chars[i] = char_chr(*octet);
  }

  string lbl = string_from_chararray(chars);

  *dest = domain_cons(lbl, *dest);
  return parse_domain_name(cursor, data, dest);
}
\end{lstlisting}

\end{document}