\documentclass{article}
\usepackage[tt=false]{libertine}
\usepackage{hyperref}
\usepackage{fullpage}

\title{Safe Shared Memory for Concurrent Algol}
\author{Farzaneh Derakhshan, Klaas Pruiksma, Jonathan Sterling}

\newcommand\Lang[1]{{\sffamily\bfseries{#1}}}
\newcommand\LangCA{\Lang{CA}}
\newcommand\LangMA{\Lang{MA}}
\newcommand\EOf[4]{{#1}\vdash_{#2}{#3}:{#4}}
\newcommand\MOf[4]{{#1}\vdash_{#2}{#3}\div{#4}}
\newcommand\IsProc[3]{{#1}\vdash_{#2}{#3}\ \mathsf{proc}}
\newcommand\IsAction[3]{{#1}\vdash_{#2}{#3}\ \mathsf{action}}


\begin{document}

\maketitle

\begin{abstract}
  Establishing safety properties for code that mixes concurrency with
  (shared) memory is notoriously difficult; the experience of the past
  several years suggests that restricted languages can provide a
  suitable means to achieve general safety results without sacrificing
  performance. Languages like \Lang{Rust} and \Lang{Go} are a step in
  the right direction, but are not currently backed by a precise
  semantics; we will attempt to thoroughly develop the theory and
  practice of a small Algol-family language featuring concurrency and
  shared memory, establishing several crucial safety properties at the
  language level.
\end{abstract}

\section{Background on programming langauges}

Farzaneh will write some general material here.

\section{Modernized and Concurrent Algol}

Our starting point is the language Concurrent Algol (\LangCA) defined
in Chapter 40 of Harper's \emph{Practical Foundations for Programming
  Languages}~\cite{harper:2016}. \LangCA{} is itself an extension of
Harper's Modernized Algol (\LangMA).

\subsection{Algol-family languages}

Algol was one of the earliest practical imperative programming
languages, and has for many years been considered the quintessential
``clean room'' core language for the scientific study of sequential,
imperative programming~\cite{reynolds:1997}; its distinctive features
has sparked a rich and sophisticated body of developments in
mathematical semantics of programming
languages~\cite{mccusker-power:2010}.

Harper's \LangMA{} differs from traditional presentations of
Algol-family languages in that it surfaces the distinction between
variables, assignables (mutable resources) and assignable
references. In \LangMA{}, a variable is a \emph{mathematical
  variable}, i.e.\ a placeholder for a term; variables are given
meaning through substitution. On the other hand, an assignable is the
name of a location in memory; finally, assignable references are a
device for passing locations in code. By restoring the correct notion
of variable to its rightful place, Harper's presentation clarifies the
structure and semantics of Algol-family languages.

\paragraph{The Essence of \LangMA{}}

The language \LangMA{} is based fundamentally on a modal distinction
between \emph{command} and \emph{expression}. A command is a piece of
code that can ``do'' something, whereas an expression is a piece of
code which calculates a value without any side effects. This is what
sets the Algol family apart from other language families which support
imperative programming (such as \Lang{C}, \Lang{ML}, \Lang{Go},
\Lang{Rust}, etc.).

Accordingly, the static semantics of \LangMA{} are organized around
two forms of assertion:

\begin{enumerate}
\item $\EOf{\Gamma}{\Sigma}{e}{\tau}$ means that $e$ is an \emph{expression}
  of type $\tau$ mentioning variables from $\Gamma$ and assignables
  (memory location names) from $\Sigma$.
\item $\MOf{\Gamma}{\Sigma}{m}{\tau}$ means that $m$ is a
  \emph{command} of type $\tau$ mentioning variables from $\Gamma$ and
  assignables from $\Sigma$.
\end{enumerate}

An operational semantics is given for \LangMA{}, which specifies how
expressions evaluate, and how commands compute relative to a heap
(memory state).



\subsection{Concurrent Algol}

\LangCA{} is an extension of \LangMA{} with a notion of \emph{process}
and their concurrent composition, together with a notion of channel
and message passing. Whilst \LangMA{} had a notion of mutable store,
\LangCA{} does away with this; Harper argues that mutable store can be
eschewed because it can be simulated using message passing.

This simplifying assumption is helpful for the development of a clean
and simple semantics, but it may not be a good idea in practice. While
shared memory comes with all sorts of disadvantages, it can be very
important for the sake of performance. Scientists who engineer
programming languages which are intended to be used in practice must
account for this.

\paragraph{What does \LangCA{} add?}
\LangCA{} adds to \LangMA{}'s static semantics the following forms of
assertion:
\begin{enumerate}
\item $\IsProc{\Gamma}{\Sigma}{p}$ means that $p$ is a \emph{process}
  relative to the variables from $\Gamma$ and the channel names from
  $\Sigma$.
\item $\IsAction{\Gamma}{\Sigma}{\alpha}$ means that $\alpha$ is an
  \emph{action} relative to $\Gamma$, $\Sigma$.
\end{enumerate}

Processes are inductively defined as the \emph{idle process}, the
atomic process $\mathtt{run}(m)$ (which executes the command $m$), and
the concurrent composition of two processes $p_1\otimes p_2$, and
finally the process $\nu\alpha.p$ which generates a fresh
communication channel and proceeds with the process $p$.

Whereas the operational semantics of \LangMA{} included a pure
transition system for expressions and a reduction semantics for
commands relative to a heap or memory state, \LangCA{} adds to this a
\emph{labelled transition system} which describes the evolution of a
system of concurrent processes. Actions $\alpha$ are the labels of
this transition system, and express the ways in which a process can
evolve; actions include querying and signalling along channels.

\paragraph{What do we hope to add to \LangCA{}?}
We hope to add primitives for shared memory to \LangCA{} and show that
our semantics is simulated by the pure message-passing implementation
of shared resources from \LangCA{}. Then, we will extend the critical
safety proofs (for instance, progress for well-typed processes) over
our extended language.

\section{Our Proposal}

We plan to create a language building on the ideas of \LangCA{} which has primitive concepts for concurrency, shared memory, and locking.
Our intent is that this language will disallow writing programs with certain classes of errors ranging from the relatively minor (For example, memory leaks or multiple acquisition of the same lock) to more complex ones including data races (potentially with some restrictions) and deadlocks.

Our 75\% goal is to develop this language and provide evidence that it is both practically usable and avoids errors by writing some sample systems.

Our 100\% goal is to strengthen our justification that the language avoids errors by providing proofs (possibly with some conditions) that those errors cannot occur in a well-typed program.
While we believe that some of these proofs will be easy to complete, we believe that those for more complex classes of errors will lead us to revise the language in order to get them to work or to work with fewer side conditions, justifying the gap between this and the 75\% goal.

Finally, our 125\% goal is to increase ease-of-use of this language by creating a simplified form for use in writing programs which can then be processed into the more expressive but more complex base language.
Since many of the conditions surrounding the use of shared memory (or just memory in general) and locks can be inferred from the code as written, we believe that it should be feasible to leave large parts of the memory management and lock management up to an elaboration step that preprocesses the simple program to make memory and lock management explicit.


\bibliographystyle{abbrv}
\bibliography{bibtex-references/refs}

\end{document}