\documentclass{article}
\usepackage[tt=false]{libertine}
\usepackage{mathpartir}
\usepackage{amsthm}
\usepackage[colorlinks=true]{hyperref}
\usepackage{bytefield}
\usepackage{fullpage}
\usepackage{listings}
\usepackage{microtype}

\lstset{language=C}
\lstset{basicstyle=\ttfamily, columns=flexible}


\usepackage{sectsty}
\allsectionsfont{\sffamily}

\title{Implementing DNS in C0}
\author{Farzaneh Derakhshan \and Klaas Pruiksma \and Jonathan Sterling}

\newtheorem{theorem}{Theorem}[section]
\newtheorem{corollary}{Corollary}[theorem]
\newtheorem{lemma}[theorem]{Lemma}

\newcommand\Kwd[1]{{\sffamily\bfseries{#1}}}
\newcommand\EOf[4]{{#1}\vdash_{#2}{#3}:{#4}}
\newcommand\MOf[4]{{#1}\vdash_{#2}{#3}\div{#4}}
\newcommand\IsProc[3]{{#1}\vdash_{#2}{#3}\ \mathsf{proc}}
\newcommand\IsAction[3]{{#1}\vdash_{#2}{#3}\ \mathsf{action}}


\begin{document}

\maketitle

\begin{abstract}
  Please write an abstract.
\end{abstract}

\section{Background and Related Work}

\subsection{What is DNS?}

Farzaneh

\subsection{C0 and Concurrent C0}\label{sec:c0}

The \Kwd{C0} programming language is a restricted subset of \Kwd{C}
that provides implicit memory safety to
programmers~\cite{pfenning:c0-reference}. Restrictions on pointer
arithmetic and casting, for example, allow the language to detect and
prevent any attempt to access an array out of its bounds. Furthermore,
there is no need for error-prone manual memory management, since
\Kwd{C0} is itself garbage-collected. Further safety conditions can be
enforced with contracts, which are a class of assertions including
loop invariants, pre- and post-conditions, and dynamically checkable
correctness conditions for programs.

\Kwd{Concurrent C0} (\Kwd{CC0}) is an extension of \Kwd{C0} that
offers concurrency based on message passing by providing processes as
the units of concurrent execution and channels as their message
buffers~\cite{willsey-prabhu-pfenning:2016}. Concurrent processes
communicate over channels using a protocol that is disciplined by
session types~\cite{caires-pfenning-toninho:2013,
  balzer-pfenning:2017}. In this protocol, each channel is typed such
that it can either send or receive a message and, thus, it
deliberately models the client/provider relation.

However, the protocol is also able to characterize processes that are
not completely sequential, and branch into different sequences as they
encounter choices. This typing discipline also guarantees that data is
sent and received in the correct order and by the correct processes
(an invariant called \emph{session fidelity}), and that the
communication is always one-to-one.



\subsection{Related Work}

\paragraph{The \Kwd{Fox} Project}
An earlier example of an implementation of the DNS protocol in a safe
and formally specified language is given by the \Kwd{Fox} project,
undertaken at Carnegie Mellon University in the
1990s~\cite{biagioni-harper-lee-milnes:1994,
  biagioni-harper-lee:2001}. Their principal artifact was a library
called \Kwd{FoxNet} which implemented an array of low-level network
infrastructure, including \Kwd{TCP}, \Kwd{IP}, \Kwd{ARP} and \Kwd{DNS}
in the \Kwd{Standard ML} language.

Unlike our development, \Kwd{FoxNet} had a highly modular character
which was enabled by the host language's sophisticated module system
(at the time, the only language to support signatures, structures and
generative functors, which are crucial for safe abstraction).

\paragraph{Other implementations} More recently, safe languages like
\Kwd{Rust} and \Kwd{Go} have been gaining popularity. Both languages
have hosted implementations of the \Kwd{DNS}
protocol~\cite{github:trust-dns,github:miekg-dns}


\section{Our Project}\label{sec:our-project}

Our original project proposal was to develop a language for safe
concurrent programming with precise formal guarantees; however, the
negotiation process led us to choose a completely unrelated project,
which we describe in this section. We will be developing an
implementation of the DNS protocol in the memory-safe \Kwd{C0}
language (see \S\ref{sec:c0}); we have divided our work into two
principal components.

\subsection{Network interface}\label{sec:network-interface}

In its current state, \Kwd{C0} has a very limited standard library. In particular, while there are libraries for input and output to files and to console, there is no networking library. It may be possible to implement networking using file I/O, but we do not believe that this is reasonable.

As such, we will write a networking library for \Kwd{C0}. Due to limitations of the language, however, this library cannot be written entirely in \Kwd{C0} itself. Instead, we will write a collection of small functions in \Kwd{C} which act as an interface between \Kwd{C0} code and the \Kwd{C} sockets library. On top of this, we can then write the \Kwd{C0} portion of the library, which will provide a higher-level interface for network communication than that provided by the \Kwd{C} portion of the library.

By splitting the library into these two parts, we can keep the \Kwd{C} portion of our codebase limited in scope. In particular, we plan to ensure that we never dynamically allocate memory from \Kwd{C} and that any accesses to \Kwd{C0}-allocated memory are done so safely. The former is easily checked for, simply by checking that there are no calls to malloc or similar functions, while the latter can be achieved (but less easily checked) by using a collection of functions for safe memory access provided to libraries by the \Kwd{C0} runtime. As a result of this and the small size of the \Kwd{C} portion of the codebase, we can easily check all of the \Kwd{C} code for memory safety. Combining this with \Kwd{C0}'s memory safety, we ensure that any memory safety violations occur outside of our project (e.g. in the \Kwd{C} sockets library or in the OS).

Since this project is limited in scope to \Kwd{DNS}, we will not expose the full functionality of the sockets library in our network interface. We will instead focus on the functions that create and work with datagram-based sockets, as most of the network communication needed for \Kwd{DNS} is done via the datagram protocol \Kwd{UDP}. Additionally, to avoid busy-waiting, we will provide a limited interface to the \texttt{poll} function of the \Kwd{C} standard library in much the same way. We believe that this is sufficient to implement our nameserver, but additional library functionality can be added using these methods if necessary.

\subsection{Marshalling and unmarshalling}

Unlike \Kwd{C}, the \Kwd{C0} language has only a single type of
integers, which are represented internally as 32-bit integers. The
network interface (see \S\ref{sec:network-interface}) will deal with
requests and responses formatted as arrays of 32-bit integers; it is
the responsibility of the marshalling-and-unmarshalling module to
unpack this data into structured representations of core DNS
datatypes.

\begin{figure}
  \centering
  \begin{bytefield}[bitwidth=1.5em]{16}
    \wordbox{1}{\texttt{ID}}
    \\
    \bitbox{1}{\texttt{QR}}
    \bitbox{4}{\texttt{Opcode}}
    \bitbox{1}{\texttt{AA}}
    \bitbox{1}{\texttt{TC}}
    \bitbox{1}{\texttt{RD}}
    \bitbox{1}{\texttt{RA}}
    \bitbox{3}{\texttt{Z}}
    \bitbox{4}{\texttt{RCODE}}
    \\
    \wordbox{1}{\texttt{QDCOUNT}}
    \\
    \wordbox{1}{\texttt{ANCOUNT}}
    \\
    \wordbox{1}{\texttt{NSCOUNT}}
    \\
    \wordbox{1}{\texttt{ARCOUNT}}
  \end{bytefield}
  \caption{The layout specification of message headers in
    DNS.}\label{fig:layout-message-header}
\end{figure}


\begin{figure}
  \centering
  \begin{bytefield}{16}
    \wordbox[lrt]{1}{\texttt{NAME}}\\
    \skippedwords\\
    \wordbox[lrb]{1}{}\\
    \wordbox{1}{\texttt{TYPE}}\\
    \wordbox{1}{\texttt{CLASS}}\\
    \wordbox{2}{\texttt{TTL}}\\
    \wordbox[lrt]{1}{\texttt{RDATA}}\\
    \skippedwords\\
    \wordbox[lrb]{1}{}
  \end{bytefield}
  \caption{The bit layout specification of resource records in
    DNS.}\label{fig:layout-resource-record}
\end{figure}

\begin{figure}
  \begin{lstlisting}[frame=single]
struct header {
  int id;
  int qr;
  int opcode;
  int aa;
  int tc;
  int rd;
  int ra;
  int z;
  int rcode;
  int qdcount;
  int ancount;
  int nscount;
  int arcount;
};

typedef struct header* header_t;
  \end{lstlisting}

  \begin{lstlisting}[frame=single]
struct domain_name {
  string label;
  struct domain_name* next;
};

typedef struct domain_name* domain_name_t;

struct resource_record {
  domain_name_t name;
  int type;
  int class;
  int ttl;
  int rd_length;
  int[] rdata;
};

typedef struct resource_record* resource_record_t;
  \end{lstlisting}
  \caption{Example DNS data structures in our \Kwd{C0} implementation.}\label{fig:c0-data-structures}
\end{figure}

For instance, Figure~\ref{fig:layout-message-header} shows the
bit-level layout of DNS message headers, and
Figure~\ref{fig:layout-resource-record} shows the same for resource
records; Figure~\ref{fig:c0-data-structures} shows how we code these
structures in our \Kwd{C0} implementation. To convert between these
structured representations and the sequences of words which are used
at the boundaries of the network
interface~(\S\ref{sec:network-interface}), we implement functions of
the following kind:

\begin{lstlisting}
int parse_domain_name(int cursor, int[] msg, domain_name_t* dest);
int parse_resource_record(int cursor, int[] msg, resource_record_t dest);
int parse_header(int cursor, int[] msg, header_t dest);
\end{lstlisting}

These functions take as an argument a cursor (an index in bits into
the message), an array of 32-bit words (the entire input message), and
pointer to a structure into which to write the parsed data; they
return a new cursor. It is important for the modularity of our
development that cursors represent not indices into the 32-bit cells
of the message, but rather bit-indices; this is because specific parts
of DNS data structures may only be a few bits long, so their parsers
cannot move the cursor forward a full word's length.
%
A sample of our parsing code can be found in the supplementary
appendix to this proposal (\S\ref{appendix:decompression}). Of
course, we will also implement the inverses to the above functions.

\section{Goals and Progress}

\subsection{75\%}

Our 75\% goal is to develop a basic \Kwd{DNS} nameserver in \Kwd{C0}. In order to reduce the scope of this project, we restrict the types of resource data (and by extension, the types of queries) that we handle, and we will only provide support for \Kwd{UDP}-based queries. These two restrictions will simplify implementation, but leave enough functionality that we can evaluate our implementation both for correctness and for performance relative to other DNS implementations, as long as we restrict the test queries to those we have implemented.

The bulk of this goal falls into the two components described in \S\ref{sec:our-project}, along with a small ``main'' component that uses these to actually handle queries. Currently, we have unmarshalling functions and about a third of the network library implemented. We expect that the remainder of this goal should go fairly quickly --- marshalling is no harder than unmarshalling, and the network interface is relatively straightforward to implement once we have decided on what functions are necessary and what their arguments should be.

\subsection{100\%}

To extend the above to a 100\% goal, we plan to extend the functionality of our nameserver by adding caching of requests and potentially adding to the types of resource data we support. While a nameserver without caching is not particularly useful, we believe that the portion of code most likely to contain memory errors is in some combination of reading data from the network and parsing that data --- in particular, dealing with message compression. As such, the 75\% goal is a proof of concept that demonstrates that these portions can be implemented safely, while the 100\% goal extends that proof of concept to be more useable (though still more limited than commercially available nameservers). We expect that in going from our 75\% version to the 100\% version, performance (and the scope of allowable queries) will increase and that we will be able to leave our arguments for correctness unchanged.

\subsection{125\%}

We have two potential 125\% goals, depending on the outcome of our 100\% goal. If our nameserver has good performance relative to its competitors, we will work to provide better justification for the correctness of our nameserver, ideally arriving at a proof that statically verifies that all of our memory accesses are safe. In doing this, we will be able to safely turn off bounds checking when running our nameserver, increasing performance through verifying correctness. If instead our performance is very poor (or not helped much by turning off bounds checking), we will work to optimize our nameserver based on the results of performance testing. It is difficult to say what specifically we will do, but we plan to take the general approach of using software like \texttt{perf} to determine what portions of our code are bottlenecks, and trying to make those paths faster.

\section{Evaluation Plan}

Klaas.


\nocite{rfc:1034, rfc:1035}
\bibliographystyle{abbrv}
\bibliography{bibtex-references/refs,project}

\clearpage
\appendix
\section{Decompressing Domain Names}\label{appendix:decompression}

\begin{lstlisting}
int parse_domain_name(int cursor, int[] data, domain_name_t* dest) {
  int* len = alloc(int);

  int orig_cursor = cursor;
  cursor = read_bits(cursor, data, 8, len);

  if (*len == 0) {
    // case: done
    *dest = domain_nil();
    return cursor;
  }

  if (*len >= 0xc0) {
    // case: pointer
    int* ptr = alloc(int);
    // read the remainder of the first 16 bits as a pointer to an offset in octets
    cursor = read_bits(orig_cursor + 2, data, 14, ptr);
    // read the last label of the domain name from the pointer:
    parse_domain_name(*ptr * 8, data, dest);
    return cursor;
  }

  // case: ordinary label
  // now '*len' specifies the number of octets in the label
  char[] chars = alloc_array(char, *len);
  for (int i = 0; i < *len; i++) {
    int* octet = alloc(int);
    cursor = read_bits(cursor, data, 8, octet);
    chars[i] = char_chr(*octet);
  }

  string lbl = string_from_chararray(chars);

  *dest = domain_cons(lbl, *dest);
  return parse_domain_name(cursor, data, dest);
}
\end{lstlisting}

\end{document}
